%%%% CAPÍTULO 1 - INTRODUÇÃO
%%
%% Deve apresentar uma visão global da pesquisa, incluindo: breve histórico, importância e justificativa da escolha do tema,
%% delimitações do assunto, formulação de hipóteses e objetivos da pesquisa e estrutura do trabalho.

%% Título e rótulo de capítulo (rótulos não devem conter caracteres especiais, acentuados ou cedilha)
\chapter{Introdução}\label{cap:introducao}
A tarefa de classificação de imagens tem adquirido relevância crescente nas últimas décadas, impulsionada pelo aumento do poder computacional e pela disponibilidade de bases de dados. Em particular, a classificação de personagens animados apresenta desafios específicos, como variações estilísticas, presença de ruído visual, iluminação não padronizada e pequenas diferenças morfológicas entre classes.

Este trabalho investiga técnicas clássicas de aprendizado de máquina aplicadas à classificação de cinco personagens da série animada The Simpsons: \textbf{Bart}, \textbf{Homer}, \textbf{Lisa}, \textbf{Maggie }e \textbf{Marge}. Para isso, uma base de imagens com estrutura de treino e validação foi organizada manualmente, com uma quantidade limitada de exemplos por classe, aproximando-se de cenários reais de bases pequenas (problema conhecido como \textit{small datasets}).

As imagens passaram por um processo de extração de características combinando \textbf{Histogram of Oriented Gradients (HOG)} e \textbf{histogramas RGB normalizados}, abordagem que busca equilibrar informações estruturais (textura, bordas) e cromáticas (cor dominante de pele, roupa, cabelo etc.), ambas importantes para distinguir personagens visualmente similares.

O pipeline de classificação inclui:

\begin{itemize}
	\item \textbf{KNN} com tuning de k e pesos
	\item \textbf{SVM} com kernel RDF
	\item \textit{Arvore de Decisão}
	\item \textbf{Random Forest}
	\item \textbf{MLP} (rede neural clássica)
	\item \textbf{Ensemble Voting} com 20 classificadores heterogêneos
\end{itemize}


Um texto curto apresentando o capítulo.

A avaliação segue as exigências acadêmicas: \textbf{validação cruzada estratificada k=10}, métricas macro e matrizes de confusão, além de um conjunto de validação separada (Valid), garantindo avaliação robusta.

Os resultados mostram que a combinação de HOG + histogramas RGB, aliada ao ensemble, oferece desempenho superior, especialmente considerando o conjunto reduzido de amostras.
