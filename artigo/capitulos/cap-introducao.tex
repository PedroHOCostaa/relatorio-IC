%%%% CAPÍTULO 1 - INTRODUÇÃO
%%
%% Deve apresentar uma visão global da pesquisa, incluindo: breve histórico, importância e justificativa da escolha do tema,
%% delimitações do assunto, formulação de hipóteses e objetivos da pesquisa e estrutura do trabalho.

%% Título e rótulo de capítulo (rótulos não devem conter caracteres especiais, acentuados ou cedilha)
\chapter{Introdução}\label{cap:introducao}
A tarefa de classificação de imagens tem adquirido relevância crescente nas últimas décadas, impulsionada pelo aumento do poder computacional e pela disponibilidade de bases de dados. Em particular, a classificação de personagens animados apresenta desafios específicos, como variações estilísticas, presença de ruído visual, iluminação não padronizada e pequenas diferenças morfológicas entre classes.

Este trabalho investiga técnicas clássicas de aprendizado de máquina aplicadas à classificação de cinco personagens da série animada The Simpsons: \textbf{Bart}, \textbf{Homer}, \textbf{Lisa}, \textbf{Maggie }e \textbf{Marge}. Para isso utilizamos a base de dados disponibilizada pelo professor, possui imagens para treino e teste, com uma quantidade limitada de exemplos por classe, aproximando-se de cenários reais de bases pequenas (problema conhecido como \textit{small datasets}).

Para decidirmos quais características utilizar, e de qual maneira extrai-las, realizamos alguns testes, treinando e medindo a acurácia conseguida utilizando diferentes grupos de características, para assim decidirmos qual utilizar na versão final do código desenvolvido. A característica utilizada foi o Histograma \gls{rgb}.

Os modelos que foram utilizados na aplicação foram os seguintes:

\begin{itemize}
	\item \textbf{KNN} com tuning de k e pesos
	\item \textbf{SVM} com kernel RDF
	\item \textit{Arvore de Decisão}
	\item \textbf{Random Forest}
	\item \textbf{MLP} (rede neural clássica)
	\item \textbf{Ensemble Voting} com 20 classificadores heterogêneos
\end{itemize}

A avaliação segue as exigências acadêmicas: \textbf{validação cruzada estratificada k=10}, métricas macro e matrizes de confusão, além de um conjunto de treino separado, garantindo avaliação robusta.

O trabalho realizado demonstra que utilizando somente o Histograma \gls{rgb} é possível conseguir resultados interessantes para a classificação de qual personagem está na imagem. E o modelo que conseguiu o melhor resultado com a estratégia abordada foi o \textit{Random Forest}.