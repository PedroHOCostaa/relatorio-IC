%%%% CAPÍTULO 3 - MATERIAL E MÉTODOS (PODE SER OUTRO TÍTULO DE ACORDO COM O TRABALHO REALIZADO)

\chapter{Conclusão}\label{cap:conclusao}

Este trabalho propôs resolver o seguinte problema proposto pelo professor: Desenvolver um código que, utilizando uma base de dados dos Simpsons, separada em um conjunto de teste e um de treino, extrair desta base de dados características, para utilizarmos estas características para treinar modelos de \gls{ia}, dentro deles um método de combinação estática de classificadores.

Ao realizarmos o experimento, conseguimos que a melhor característica encontrada foi o \gls{rgb} e que o modelo que conseguiu o melhor resultado foi o \textit{Random Forest}.

\section{Trabalhos Futuros}
Sugerimos que para obter resultados melhores, utilizem alguma técnica de segmentação de imagens que produza características numéricas, assim utilizando não somente as cores como tambem os "contornos" para classificarmos os personagens.